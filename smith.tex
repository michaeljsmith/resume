% LaTeX resume using res.cls
\documentclass{res}
%\usepackage{helvetica} % uses helvetica postscript font (download helvetica.sty)
%\usepackage{newcent}   % uses new century schoolbook postscript font  
\setlength{\topmargin}{-0.6in}  % Start text higher on the page 
\setlength{\textheight}{9.8in}  % increase textheight to fit more on a page
\setlength{\headsep}{0.2in}     % space between header and text
\setlength{\headheight}{12pt}   % make room for header
\usepackage{fancyhdr}  % use fancyhdr package to get 2-line header
\renewcommand{\headrulewidth}{0pt} % suppress line drawn by default by fancyhdr
\lhead{\hspace*{-\sectionwidth}Michael Smith} % force lhead all the way left
\rhead{Page \thepage}  % put page number at right
\cfoot{}  % the footer is empty
\pagestyle{fancy} % set pagestyle for the document

\begin{document} 
\thispagestyle{empty} % this page does not have a header
\name{Michael Smith}
\address{120 Rose St\\
Fitzroy, VIC 3065\\
+61 (0) 421 487 249}

\begin{resume}
\vspace{0.1in}
 
\section{Employment History} 
\vspace{0.1in} 
	{\bf Programmer} \\
	Endgame Studios \\
	July 2009 - Present \\

	Work on application for playing 3D-spatialized sound over headphones.
	One application was used for calibration and generation of transfer functions,
	which need to be computed individually for each user/headphone pair. Work also
	required to interface with kernel level devices in Windows to intercept 7.1
	sound and render it over headphones.

	Development of small-scale retail software for IPhone and Android platforms,
	including games and musical apps. Working in small teams or alone on all
	aspects of project development.

	Development of games for Nintendo DS. Collision and physics behaviour for 2D
	game. Image analysis for photos taken using console camera.

	{\bf Software Development Consultant} \\
	Crytek \\
	December 2008 - July 2009 \\

	Primary responsibility involved creation of game asset pipeline
	allowing use of 3D graphics assets created in Softimage XSI and Autodesk Maya
	3D graphics packages in Crytek's game engine (CryEngine 2). Emphasis was on
	a robust and intuitive experience for artists in creating game assets.

	{\bf Lead Tools Programmer} \\
	Crytek \\
	December 2007 - December 2008 \\

	Role involved creation and management of the tools programming group as
	part of the R\&D department. This group was responsible for all proprietary
	software used in the creation of software using the Crytek engine.

	Position also included a strong element of technical support, both in fielding
	questions from the technical support team regarding tools, and in supporting
	in-house developers in creating games.

	{\bf Senior R\&D Programmer} \\
	Crytek \\
	November 2005 - November 2007 \\

	Creation and maintenance of tools used in creation of games, both by in-house
	developers and licensees of game engine.

	Development of facial animation package, allowing users to combine rich vertex
	morph targets and skeletal animations into rich layered expressions using
	spline interpolation. Animations could be created using a graphical editor and
	viewed in real-time, even as characters were being used in-game. Features were
	included for smoothing and cleaning motion-captured data.

	Performance analysis and optimization of game and engine code. Use of
	profiling software for finding and solving performance bottlenecks in
	speed-critical code.

	Memory optimization of various areas of game and engine code.

	Creation and maintenance of support tools for importing game assets from
	modelling packages, in particular Autodesk 3DS Max. Assets included 3D models,
	both static and skinned characters, as well as animation data.

	{\bf Programmer} \\
	THQ/BlueTongue \\
	June 2005 - September 2005 \\

	Worked on game code for Gamecube and Playstation 2. Enemy pathfinding and boss
	enemy behaviour. Also worked on Tutorial mode for game, instructing users on
	how to play the game.

	{\bf Programmer} \\
	Torus Games \\
	November 1999 - December 2003 \\

	Worked on 14 game titles for various platforms. All aspects of game creation
	covered, from asset management software to interfacing with console graphics
	hardware to high-level game content code.

	Creation of 3D software renderer for embedded platforms. Performance was
	extremely critical - majority of executed code was written in ARM assembly.
	Engine was based on a Binary Space Partitioning (BSP) tree (as used in games like
	Quake). Engine included texture mapping, realistic lighting via lightmaps
	(pre-computed and dynamic) and animated characters. Memory and CPU were both
	highly restricted.

\section{Education}
\vspace{0.1in} 

    {\bf Bachelor of Computer Science and Computing (Hons)} \\
	1999 - 2004 Monash University, Melbourne, Australia

	Honours Thesis: Lossless Image Coding using Probability Distribution Blending

	Standard lossless image encoders employ arithmetic prediction algorithms to
	reduce the entropy of encoded symbols. Many combine the outputs of multiple
	predictors to produce a single predicted value. This project instead combined
	the probability distributions created by each predictor into a single,
	possibly multi-modal distribution, which is then fed directly into the entropy
	coder.

	A subtext of the project was the use of the most successful model as a
	high-level description of the image, as an attempt to provide a useful
	segmentation of the image into logical partitions.
	 
\section{Skills and Focus} 
\vspace{0.2in} 

I have a broad set of software development skills, and consider myself strong
in algorithm/data structure design, architecture and debugging. I have a good
understanding of multithreading design and scalability.

I consider software development to be all about judgement. There are no rules
that one can follow that work in every case - making the correct call on
architecture and design is something that requires a great deal of experience
to be able to balance all the competing requirements. Similarly when debugging
it is important to figure out the best way to answer the question as quickly
as possible. I think it's actually science on a small scale - hypothesizing,
developing a means of testing that hypothesis and then repeating many times a
day.

\section{Interests}

I found my experience working at Crytek in Germany to be quite enlightening -
people from all over the world were employed there, and it was really
eye-opening to be exposed to such a wide variety of European and world
cultures within one place. I'm always driven to develop a stronger
understanding of things. For instance, what led certain cultures to develop
certain traits relative to others? Does it come down to historical factors? Is
it geographical? Is it the weather?  In the same way I'm always interested in
the roots of things, words, ideas.  For this reason I have a keen interest in
history (although I have a terrible head for dates).

\section{References}

References available on demand.

\end{resume}
\end{document}
